% This template was designed by Ruben De Smet. Visit his website at https://gitlab.com/rubdos/texlive-vub

\documentclass{article}
\newcommand{\unit}[1]{\ensuremath{\, \mathrm{#1}}}
\usepackage{gensymb}
\usepackage[english]{babel}
\usepackage{amsmath}
\usepackage{calc}
\usepackage{vub}
\usepackage{chemfig}
\usepackage{multicol}
\usepackage{caption}
\usepackage{pgfplots}
\usepackage{float}
\usepackage{tikz}
\usetikzlibrary{shapes.geometric, arrows}

\title{e-commerce Project}
\subtitle{Open Information Systems}
\author{Ardavan Khalij, Amadou Sarjo Jallow, Brenda Ordoñez Lujan, Florent Nicolas J Grimau, and Milan Pavle Ilic}
\faculty{1st master computer science}
\promotors{Academic year: 2021 - 2022\\ Student number: Milan 0545802\\ VUB-email-address: milan.pavle.ili@vub.be}
\pretitle{}
\date{26 December 2022}

\usepackage{pgfplots}
\def\triangleH{27.7mm}
\renewcommand{\contentsname}{Contents}
\begin{document}
\maketitle
\tableofcontents
\newpage
\section{Introduction}
The goal of this project is to design and implement an e-commerce platform. The backbone of this platform is an information system. This system consists out of 5 modules: Products, Warehouses, Costumers, Suppliers and Delivery Services.

In the beginning, each person was assigned to one module. Firstly we started thinking about the information needs of the database users by creating features and an ER diagram(Entity-Relation diagram) for a module. Once this was achieved, each person created a database with queries, based on the created ER diagram, in standard SQL. We then created one information system by combining the SQL databases. After that, everyone created an ontology and mapping for their module.

\section{Ontology}


\section{Mapping}


\section{Queries}


\section{Discussion}


\section{References}


\end{document}
