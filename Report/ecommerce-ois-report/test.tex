% This template was designed by Ruben De Smet. Visit his website at https://gitlab.com/rubdos/texlive-vub

\documentclass{article}
\newcommand{\unit}[1]{\ensuremath{\, \mathrm{#1}}}
\usepackage{gensymb}
\usepackage[english]{babel}
\usepackage{amsmath}
\usepackage{calc}
\usepackage{vub}
\usepackage{chemfig}
\usepackage{multicol}
\usepackage{caption}
\usepackage{pgfplots}
\usepackage{float}
\usepackage{tikz}
\usetikzlibrary{shapes.geometric, arrows}

\title{e-commerce Project}
\subtitle{Open Information Systems}
\author{Ardavan Khalij, Amadou Sarjo Jallow, Brenda Ordoñez Lujan, Florent Nicolas J Grimau, and Milan Pavle Ili\'c}
\faculty{1st master computer science}
\promotors{Academic year: 2021 - 2022\\ Student number: Brenda 0571129, Milan 0545802\\ VUB-email-address: brenda.ordonez.lujan@vub.be,\\ milan.pavle.ili@vub.be}
\pretitle{}
\date{26 December 2021}

\usepackage{pgfplots}
\def\triangleH{27.7mm}
\renewcommand{\contentsname}{Contents}
\begin{document}
\maketitle
\tableofcontents
\newpage
\section{Introduction}
The goal of this project is to design and implement an e-commerce platform. The backbone of this platform is an information system. This system consists out of 5 modules: Products, Warehouses, Costumers, Suppliers and Delivery Services. \\

Each person was assigned to one module. Firstly we started thinking about the information needs of the database users by creating features and an ER diagram\footnote{Entity-Relation diagram} for a module. Once this was achieved, each person created a database with queries, based on the created ER diagram, in standard SQL\footnote{Structured Query Language}. We then created one information system by combining the SQL databases. After that, everyone created an ontology, using Protoge, for their module. Then we combined the ontologies to create one information system. Then everyone created a mapping using the R2RML\footnote{RDB to RDF Mapping Language} mapping language. Again, we combined our mappings to generate one mapping that represented the whole system. Finally everyone created one SPARQL\footnote{SPARQL Protocol and RDF Query Language} query, based on one of their features. This SPARQL query was tested using the Ontop system. Ontop translates SPARQL queries into SQL queries, and relies on R2RML mappings \cite{OntopVK}.

\subsection{Modules}
As metioned before, this project consists of 5 modules:
\begin{enumerate}
  \item The module Product is about the physical product itself and the properties it has.

  \item The module Warehouses is primarily about the management of an inventory.

  \item The module Costumers is concerned with reviews, payments, and other things regarding an online costumer.

  \item The module Suppliers is primarily about a supplier that supplies to a warehouse.

  \item The module Delivery services is about delivering the product from a warehouse to a costumer.
\end{enumerate}

\subsection{Features and data}

Product features:
\begin{enumerate}
  \item
\end{enumerate}

\noindent Warehouses features:
\begin{enumerate}
  \item Product quantity: An employee of the inventory management wants to find out how much of a certain product there is left in stock. An online costumer wants to know if the product is available for purchase.
  \underline{Data used:} amount of stock of a certain product.
  \item Warehouse shipments: An employee wants to know how many shipments will arrive/departure on a certain day. An accountant wants to keep track of all the arriving and departing products.
  \underline{Data used:} all the shipments, the products of each shipment, the delivery services of each shipment.
  \item Shipment products: A warehouse employee checks if an outgoing/incoming shipment has all the packages. A costumer wants to check if his product has already left the warehouse.
  \underline{Data used:} all the products that a shipment contains.
  \item Shipment delivery time: The employee of a warehouse wants to know when a shipment arrives. A costumer wants to know how long it will take for his package to arrive.
  \underline{Data used:} details about the delivery services and the destination and departure address of the shipment.
  \item Business information: An employee of a certain business wants to contact another business. A business needs to deliver to another business.
  \underline{Data used:} information about a warehouse or delivery service.
\end{enumerate}

\noindent Costumers features:
\begin{enumerate}
  \item
\end{enumerate}

\noindent Suppliers features:
\begin{enumerate}
  \item Response time per supplier: An employee of the logistics area needs the average response time of an order to know when he needs to fill his stock if required. A manager of the logistics area needs it to know the performance of the supplier and to be able to follow up on it. \underline{Data used:} Expressed in days and hours, it is calculated with the following operation. Date the order was placed - Date the order was delivered.
  \item The supplier that gives more facilities: A manager of the logistics area needs it to make the decision to reduce costs if it’s possible or choose between suppliers in the future. \underline{Data used:} For this feature, the count number and the average of the discount for prompt payment per supplier is needed as a field expressed in percentages, the discount is entered when the order is being placed, if the supplier offers it.
  \item Maximum payment flexibility: An employee of the logistics area needs it to make budget plans. A manager of the logistics area needs it to make decisions such as changing the payday if necessary or choose between suppliers in the future. \underline{Data used:} Expressed in days. It is the maximum of the limit days to pay that gives the supplier.
  \item Minimum cost of the most product requested: An employee of the logistics area needs it to calculate the necessary budget and predict the days that needs to place orders. A manager of the logistics area needs it to compare with other prices on the market and make decisions between new suppliers. \underline{Data used:} Expressed in euros, it is minimum cost of the wholesale price of the product that is being ordered the most.
  \item Higher and lower delivery costs per supplier: An employee in the logistics area can reduce transportation costs from that value. Analyze whether it is convenient to have a transportation service and present it to the manager as an alternative in the future. It is also an important factor in choosing between providers. \underline{Data used:} Expressed in euros, it is maximum cost of the delivery cost per supplier.
\end{enumerate}

\noindent Delivery services features:
\begin{enumerate}
  \item Real-time package Tracking: A costumer wants to track their order in real time. \underline{Data used:} The API key linked to the transport company handling the package the user wants to track.
  \item Delivery Time Estimation by packet: A delivery service wants to give the costumer an idea when the package will arive. \underline{Data used:} The address registered in the packet’s order, the packet’s weight and size, the amount of trucks available in the company able to deliver a packet to the given address with the given weight and size and the location of the warehouse where the packet is gathered.
  \item Delivery Price Estimation: Delivery service wants to request an extra fee to the costumer. \underline{Data used:} The address registered in the packet’s order, the packet’s weight and size, the price per km of the company able to deliver the package (As explained in the previous feature).
  \item Average amount of delivery trip made by each delivery companies in all the warehouses: A delivery service wants to track with which companies it is worth having delivery contracts, based on the amount of delivery trips they make for the company. \underline{Data used:} he amount of delivery trip that occurred in every warehouse in the last X months sorted by the vehicle’s company.
  \item Average Packet loss during delivery trips sorted by company: A delivery service wants to increase its trust. \underline{Data used:} The average status of the packets in the deliveries of company X.
\end{enumerate}


\subsection{Work division}
Ardavan Khalij: Costumer database and queries, Costumer ontology, Costumer mapping, SPARQL query, ...
\\

\noindent Amadou Sarjo Jallow: Product database and queries, Product ontology, Product mapping, SPARQL query, ...
\\

\noindent Brenda Ordoñez Lujan: Supplier database and queries, Supplier ontology, Supplier mapping, SPARQL query, ...
\\

\noindent Florent Nicolas J Grimau: Delivery service database and queries, Delivery service ontology, Delivery service mapping, SPARQL query, ...
\\

\noindent Milan Pavle Ilic: Warehouses database and queries, Warehouse ontology, Warehouse mapping, SPARQL query, Report introduction and template.



\section{Ontology}


\section{Mapping}


\section{Queries}


\section{Discussion}



\begin{thebibliography}{99}
\bibitem{OntopVK}
Guohui Xiao, Davide Lanti, Roman Kontchakov, Sarah Komla-Ebri, Elem Güzel-Kalayci, Linfang Ding, Julien Corman, Benjamin Cogrel, Diego Calvanese, and Elena Botoeva. (2020) \emph{The Virtual Knowledge Graph System Ontop}, International Semantic Web Conference.
\bibitem{OntopSPARQL}
Diego Calvanese, Benjamin Cogrel, Sarah Komla-Ebri, Roman Kontchakov, Davide Lanti, Martin Rezk, Mariano Rodriguez-Muro, and Guohui Xiao. (2017) \emph{Ontop: Answering SPARQL Queries over Relational Databases}, Semantic Web Journal 8.3, pp. 471–487.
\end{thebibliography}






\end{document}
